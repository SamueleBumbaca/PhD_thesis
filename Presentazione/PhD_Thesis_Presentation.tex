% filepath: /home/samuelebumbaca/repositories/PhD_thesis/Presentazione/PhD_Thesis_Presentation.tex
\documentclass[aspectratio=43]{beamer}

% Theme and color scheme
\usetheme{Madrid}
\usecolortheme{default}

% Packages
\usepackage[utf8]{inputenc}
\usepackage[T1]{fontenc}
\usepackage{graphicx}
\usepackage{amsmath}
\usepackage{amsfonts}
\usepackage{amssymb}
\usepackage{hyperref}
\usepackage{multicol}
\usepackage{tikz}
\usepackage{xcolor}

% Custom colors
\definecolor{uniblue}{RGB}{0,51,102}
\definecolor{highlight}{RGB}{255,153,0}

% Set theme colors
\setbeamercolor{structure}{fg=uniblue}
\setbeamercolor{frametitle}{bg=white,fg=black}

% Custom headline: white background + thin uniblue line at the bottom
\setbeamertemplate{frametitle}{
  \nointerlineskip
  \begin{beamercolorbox}[wd=\paperwidth,ht=2.5ex,dp=1ex,left]{frametitle}
    \vspace*{0.5ex}\strut\insertframetitle\strut
  \end{beamercolorbox}
  {\color{uniblue}\rule{\paperwidth}{0.7mm}}
}

% Remove navigation symbols
\setbeamertemplate{navigation symbols}{}

% Remove footline
\setbeamertemplate{footline}{}

% Custom title page
\setbeamertemplate{title page}{
    \begin{picture}(0,0)
        \put(25,-220){\includegraphics[width=0.8\paperwidth]{Imgs/Loghi.png}}
    \end{picture}
    \vfill
    \centering
    \begin{beamercolorbox}[sep=8pt,center]{title}
        \usebeamerfont{title}\inserttitle\par%
        \ifx\insertsubtitle\@empty%
        \else%
            \vskip0.25em%
            {\usebeamerfont{subtitle}\usebeamercolor[fg]{subtitle}\insertsubtitle\par}%
        \fi%
    \end{beamercolorbox}%
    \vskip1em\par
    \begin{beamercolorbox}[sep=8pt,center]{author}
        \usebeamerfont{author}\insertauthor
    \end{beamercolorbox}
    \begin{beamercolorbox}[sep=8pt,center]{institute}
        \usebeamerfont{institute}\insertinstitute
    \end{beamercolorbox}
    \begin{beamercolorbox}[sep=8pt,center]{date}
        \usebeamerfont{date}\insertdate
    \end{beamercolorbox}\vskip0.5em
    \vfill
}

% Document information
\title{Geomatic Techniques to Support Phytosanitary Products Tests whithin the EPPO Standard Framework}
% \subtitle{Geomatics Technologies for Enhanced Plant Protection Product Registration}
\author{Samuele Bumbaca}
\institute{University of Turin}
\date{July 17, 2025}

\begin{document}

% Title slide
\begin{frame}
    \titlepage
\end{frame}

% Slide 2: Presentation Outline
\begin{frame}
    \frametitle{Presentation Structure (40 minutes)}
    
    \begin{enumerate}
        \item \textbf{Introduction \& Background} (20 minutes)
        \begin{itemize}
            \item Research problem and motivation
            \item Theoretical framework  
            \item Methodology overview
        \end{itemize}
        
        \item \textbf{Three Case Studies} (18 minutes total)
        \begin{itemize}
            \item Plant Counting (6 minutes)
            \item Phytotoxicity Scoring (6 minutes)
            \item Anomaly Detection (6 minutes)
        \end{itemize}
        
        \item \textbf{Conclusions \& Future Work} (2 minutes)
    \end{enumerate}
\end{frame}

% Slide 3: The Problem
\begin{frame}
    \frametitle{Current Limitations in Agricultural Statistics}
    
    \begin{block}{Traditional Approach Issues:}
        \begin{itemize}
            \item \textbf{Human-dependent blocking}: Environmental variability assessment relies on experimenter experience
            \item \textbf{A priori identification}: Must identify variance sources BEFORE data collection
            \item \textbf{Limited statistical power}: When assumptions fail, must resort to non-parametric tests
            \item \textbf{Regulatory requirements}: EPPO standards demand R² > 0.85 performance
        \end{itemize}
    \end{block}
    
    \begin{alertblock}{The Challenge:}
        \textit{How can we capture environmental variability mathematically rather than through human judgment?}
    \end{alertblock}
\end{frame}

% Slide 4: Research Gap
\begin{frame}
    \frametitle{The Missing Link: Spatial Coordinates}
    
    \begin{columns}
        \begin{column}{0.5\textwidth}
            \begin{block}{Geostatistical Methods Advantages:}
                \begin{itemize}
                    \item[\textcolor{green}{\checkmark}] \textbf{Mathematical modeling} of environmental variability
                    \item[\textcolor{green}{\checkmark}] \textbf{Post-hoc analysis} - no need for prior knowledge
                    \item[\textcolor{green}{\checkmark}] \textbf{Superior performance} in handling spatial heterogeneity
                    \item[\textcolor{green}{\checkmark}] \textbf{EPPO recognized} approach
                \end{itemize}
            \end{block}
        \end{column}
        
        \begin{column}{0.5\textwidth}
            \begin{alertblock}{Current Barrier:}
                \begin{itemize}
                    \item[\textcolor{red}{\times}] \textbf{Requires spatially referenced observations}
                    \item[\textcolor{red}{\times}] \textbf{Traditional manual assessments lack coordinates}
                    \item[\textcolor{red}{\times}] \textbf{Implementation gap} in practical field trials
                \end{itemize}
            \end{alertblock}
        \end{column}
    \end{columns}
\end{frame}

% Slide 5: Research Question
\begin{frame}
    \frametitle{Central Research Question}
    
    \begin{exampleblock}{}
        \large
        \textbf{Can geomatics technologies provide spatially referenced observations that enable geostatistical analysis within EPPO-compliant Plant Protection Product trials?}
    \end{exampleblock}
    
    \vspace{1em}
    
    \begin{block}{Specific Objectives:}
        \begin{enumerate}
            \item Establish minimum dataset requirements for digital data collection
            \item Demonstrate feasibility across all EPPO variable types
            \item Validate performance against traditional methods
            \item Provide practical implementation guidelines
        \end{enumerate}
    \end{block}
\end{frame}

% Slide 6: EPPO Standards Framework
\begin{frame}
    \frametitle{European Plant Protection Organization (EPPO)}
    
    \begin{block}{Key Standards:}
        \begin{itemize}
            \item \textbf{PP 1/152(4)}: Design and analysis of efficacy evaluation trials
            \item \textbf{PP 1/333(1)}: Digital technology adoption guidelines
        \end{itemize}
    \end{block}
    
    \begin{block}{Variable Types in EPPO Assessments:}
        \begin{enumerate}
            \item \textbf{Continuous/Discrete}: Plant counts, measurements
            \item \textbf{Ordinal}: Severity scales (0-100\%), damage ratings
            \item \textbf{Binary/Nominal}: Healthy/diseased, disease classification
        \end{enumerate}
    \end{block}
    
    \begin{alertblock}{Benchmark: R² > 0.85 compared to manual assessment}
    \end{alertblock}
\end{frame}

% Continue with more slides...
% For brevity, I'll show a few more key slides and provide the structure

% Slide 7: Plant Protection Products Context
\begin{frame}
    \frametitle{PPP Development \& Regulation}
    
    \begin{columns}
        \begin{column}{0.5\textwidth}
            \begin{block}{PPP Categories:}
                \begin{itemize}
                    \item Fungicides
                    \item Insecticides
                    \item Herbicides
                    \item Plant growth regulators
                    \item Acaricides
                    \item Nematicides
                \end{itemize}
            \end{block}
        \end{column}
        
        \begin{column}{0.5\textwidth}
            \begin{block}{Critical Evaluation Needs:}
                \begin{itemize}
                    \item \textbf{Efficacy}: Does it work?
                    \item \textbf{Selectivity}: Is it safe for crops?
                    \item \textbf{Environmental impact}: Side effects?
                \end{itemize}
            \end{block}
        \end{column}
    \end{columns}
\end{frame}

% Slide with image example
\begin{frame}
    \frametitle{Technical Arsenal}
    
    \begin{columns}
        \begin{column}{0.6\textwidth}
            \begin{block}{Core Technologies:}
                \begin{itemize}
                    \item \textbf{Photogrammetry}: 3D model generation from 2D images
                    \item \textbf{Spectral Imaging}: Multi/hyperspectral sensors
                    \item \textbf{Machine Learning}: Object detection, classification, regression
                    \item \textbf{GNSS/UAV}: Precise spatial positioning
                \end{itemize}
            \end{block}
        \end{column}
        
        \begin{column}{0.4\textwidth}
            % Example of how to include an image
            % \includegraphics[width=\textwidth]{images/technology_overview.png}
            \begin{center}
                \textit{[Technology diagram would go here]}
            \end{center}
        \end{column}
    \end{columns}
\end{frame}

% Study 1 Section
\section{Study 1: Plant Counting}

\begin{frame}
    \frametitle{Study 1: Automated Plant Counting}
    
    \begin{columns}
        \begin{column}{0.6\textwidth}
            \begin{block}{Problem Statement:}
                Manual plant counting is:
                \begin{itemize}
                    \item \textbf{Time-consuming}: Hours per plot
                    \item \textbf{Subjective}: Inter-observer variability
                    \item \textbf{Error-prone}: Missed or double-counted plants
                    \item \textbf{Non-spatial}: No coordinate information
                \end{itemize}
            \end{block}
            
            \begin{exampleblock}{Solution Approach:}
                \begin{itemize}
                    \item UAV photogrammetry
                    \item Deep learning object detection
                    \item Automatic spatial referencing
                    \item R² > 0.85 validation
                \end{itemize}
            \end{exampleblock}
        \end{column}
        
        \begin{column}{0.4\textwidth}
            % Image placeholder
            \begin{center}
                \textit{[Plant counting example image]}
            \end{center}
        \end{column}
    \end{columns}
\end{frame}

% Results slide with table
\begin{frame}
    \frametitle{Minimum Dataset Findings}
    
    \begin{table}
        \centering
        \begin{tabular}{|l|c|c|}
        \hline
        \textbf{Architecture} & \textbf{Images Needed} & \textbf{R²} \\
        \hline
        RT-DETR (Transformer-mixed) & 60 & 0.89 \\
        YOLOv8 (CNN) & 110 & 0.87 \\
        YOLOv5 (CNN) & 130 & 0.86 \\
        Few-shot models & N/A & < 0.85 \\
        Zero-shot models & N/A & < 0.85 \\
        \hline
        \end{tabular}
        \caption{Performance comparison across architectures}
    \end{table}
    
    \begin{alertblock}{Critical Finding:}
        \textbf{NO out-of-distribution trained model achieved R² > 0.85}\\
        \textit{In-domain training data is essential for agricultural applications}
    \end{alertblock}
\end{frame}

% Thank you slide
\begin{frame}
    \frametitle{Thank You}
    
    \begin{center}
        \Huge Questions \& Discussion
    \end{center}
    
    \vspace{2em}
    
    \begin{block}{Contact Information:}
        \textbf{Samuele Bumbaca}\\
        University of Turin\\
        Email: samuele.bumbaca@unito.it
    \end{block}
    
    \begin{block}{Key Publications:}
        \begin{enumerate}
            \item "On the Minimum Dataset Requirements..." - \textit{Remote Sensing} (2025)
            \item "Supporting Screening of New Plant Protection Products..." - \textit{Agronomy} (2024)
            \item "Anomaly Detection for Plant Disease Classification" - \textit{In preparation}
        \end{enumerate}
    \end{block}
\end{frame}

\end{document}