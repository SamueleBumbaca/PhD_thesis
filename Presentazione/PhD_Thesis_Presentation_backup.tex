% filepath: /home/samuelebumbaca/repositories/PhD_thesis/Presentazione/PhD_Thesis_Presentation.tex
\documentclass[aspectratio=43]{beamer}

% Theme and color scheme
\usetheme{Madrid}
\usecolortheme{default}

% Packages
\usepackage[utf8]{inputenc}
\usepackage[T1]{fontenc}
\usepackage{graphicx}
\usepackage{amsmath}
\usepackage{amsfonts}
\usepackage{amssymb}
\usepackage{hyperref}
\usepackage{multicol}
\usepackage{tikz}
\usepackage{xcolor}

% Custom colors
\definecolor{uniblue}{RGB}{0,51,102}
\definecolor{highlight}{RGB}{2            \begin{block}{\small Spatia    
    \begin{block}{\scriptsize Benefits:}
        \tiny
        \begin{itemize}
            \item \textbf{No blocking}: Spatial correlation captures variability
            \item \textbf{Post-hoc}: No a priori variance identification
            \item \textbf{Kriging}: Optimal spatial interpolation \& uncertainty maps
        \end{itemize>
    \end{block}               \begin{equation*}
                    y(s_i) = \mu + \alpha_j + f(s_i) + \varepsilon_i
                \end{equation*}
                
                \scriptsize
                Where:
                \begin{itemize}
                    \item $y(s_i)$ = response at $s_i$
                    \item $\mu$ = overall mean
                    \item $\alpha_j$ = treatment effect
                    \item $f(s_i)$ = spatial field
                    \item $\varepsilon_i$ = error
                    \item $s_i = (x_i, y_i)$
                \end{itemize}
            \end{block}
            
            \begin{exampleblock}{\scriptsize Key Advantage:}
                {\tiny Environmental variability \textbf{modeled mathematically} vs. \textbf{assumed blocking}}
            \end{exampleblock}onmental gradient colors (define globally)
\definecolor{envlow}{RGB}{255,245,235}     % Light orange
\definecolor{envmed}{RGB}{253,174,97}      % Medium orange
\definecolor{envhigh}{RGB}{215,48,31}      % Dark red

% Set theme colors
\setbeamercolor{structure}{fg=uniblue}
\setbeamercolor{frametitle}{bg=white,fg=black}

% Custom headline: white background + thin uniblue line at the bottom
\setbeamertemplate{frametitle}{
  \nointerlineskip
  \begin{beamercolorbox}[wd=\paperwidth,ht=2.5ex,dp=1ex,left]{frametitle}
    \hspace*{2em} % Move text right
    \vspace*{-1.5ex}\strut\insertframetitle\strut
  \end{beamercolorbox}
  \hspace*{-2em} % Move line right
  {\color{uniblue}\rule{\paperwidth}{0.7mm}}
}

% Remove navigation symbols
\setbeamertemplate{navigation symbols}{}

% Remove footline
\setbeamertemplate{footline}{}

% Custom title page
\setbeamertemplate{title page}{
    \begin{picture}(0,0)
        \put(25,-220){\includegraphics[width=0.8\paperwidth]{Imgs/Loghi.png}}
    \end{picture}
    \vfill
    \centering
    \begin{beamercolorbox}[sep=8pt,center]{title}
        \usebeamerfont{title}\inserttitle\par%
        \ifx\insertsubtitle\@empty%
        \else%
            \vskip0.25em%
            {\usebeamerfont{subtitle}\usebeamercolor[fg]{subtitle}\insertsubtitle\par}%
        \fi%
    \end{beamercolorbox}%
    \vskip1em\par
    \begin{beamercolorbox}[sep=8pt,center]{author}
        \usebeamerfont{author}\insertauthor
    \end{beamercolorbox}
    \begin{beamercolorbox}[sep=8pt,center]{institute}
        \usebeamerfont{institute}\insertinstitute
    \end{beamercolorbox}
    \begin{beamercolorbox}[sep=8pt,center]{date}
        \usebeamerfont{date}\insertdate
    \end{beamercolorbox}\vskip0.5em
    \vfill
}

% Document information
\title{Geomatic Techniques to Support Phytosanitary Products Tests whithin the EPPO Standard Framework}
% \subtitle{Geomatics Technologies for Enhanced Plant Protection Product Registration}
\author{Samuele Bumbaca}
\institute{University of Turin}
\date{August 28, 2025}

\begin{document}

% Title slide
\begin{frame}
    \titlepage
\end{frame}

% Slide 2: Presentation Outline
%\begin{frame}
%    \frametitle{Presentation Structure (40 minutes)}
%    
%    \begin{enumerate}
%        \item \textbf{Introduction \& Background} (20 minutes)
%        \begin{itemize}
%            \item Research problem and motivation
%            \item Theoretical framework  
%            \item Methodology overview
%        \end{itemize}
%        
%       \item \textbf{Three Case Studies} (18 minutes total)
%      \begin{itemize}
%            \item Plant Counting (6 minutes)
%            \item Phytotoxicity Scoring (6 minutes)
%            \item Anomaly Detection (6 minutes)
%        \end{itemize}
%        
%        \item \textbf{Conclusions \& Future Work} (2 minutes)
%    \end{enumerate}
%\end{frame}

% Slide 2: The Traditional Approach to Agricultural Trials
\begin{frame}
    \frametitle{The Traditional Approach to Agricultural Trials}
    
    \begin{columns}
        \begin{column}{0.65\textwidth}
            \begin{center}
                \begin{tikzpicture}[scale=0.8]
                    % Define colors for treatments
                    \definecolor{control}{RGB}{220,220,220}
                    \definecolor{tested}{RGB}{173,216,230}
                    \definecolor{reference}{RGB}{255,182,193}
                    
                    % Draw grid and plots
                    \foreach \row in {0,1,2} {
                        \foreach \col in {0,1,2} {
                            % Calculate position
                            \pgfmathsetmacro{\x}{\col*2}
                            \pgfmathsetmacro{\y}{\row*1.5}
                            
                            % Assign treatments (Latin square-like design)
                            \pgfmathsetmacro{\treatment}{int(mod(\col + \row, 3))}
                            \ifnum\treatment=0
                                \def\plotcolor{control}
                                \def\plotlabel{C}
                            \fi
                            \ifnum\treatment=1
                                \def\plotcolor{tested}
                                \def\plotlabel{T}
                            \fi
                            \ifnum\treatment=2
                                \def\plotcolor{reference}
                                \def\plotlabel{R}
                            \fi
                            
                            % Draw plot
                            \fill[\plotcolor] (\x,\y) rectangle (\x+1.5,\y+1);
                            \draw[black, thick] (\x,\y) rectangle (\x+1.5,\y+1);
                            \node at (\x+0.75,\y+0.5) {\Large\textbf{\plotlabel}};
                        }
                        % Block labels
                        \node[left] at (-0.3,\row*1.5+0.5) {\textbf{Block \pgfmathprint{int(\row+1)}}};
                    }
                    
                    % Legend
                    \node[below] at (3,-0.5) {
                        \begin{tabular}{ll}
                            \textbf{C} & Control \\
                            \textbf{T} & Tested Product \\
                            \textbf{R} & Reference Product
                        \end{tabular}
                    };
                \end{tikzpicture}
            \end{center}
        \end{column}
        
        \begin{column}{0.35\textwidth}
            \begin{block}{ANOVA Model:}
                \begin{equation*}
                    y_{ij} = \mu + \alpha_i + \beta_j + \varepsilon_{ij}
                \end{equation*}
                
                \small
                Where:
                \begin{itemize}
                    \item $y_{ij}$ = response
                    \item $\mu$ = overall mean
                    \item $\alpha_i$ = treatment effect
                    \item $\beta_j$ = block effect
                    \item $\varepsilon_{ij}$ = random error
                \end{itemize}
            \end{block}
            
            \begin{alertblock}{\tiny Note:}
                {\tiny This is the \textbf{additive model}. Modern approaches may include interaction terms: $\alpha_i \times \beta_j$}
            \end{alertblock}
        \end{column}
    \end{columns}
\end{frame}

% Slide 3: Key Assumptions of Traditional ANOVA
\begin{frame}
    \frametitle{Key Assumptions of Traditional ANOVA}
    
    \begin{block}{Statistical Assumptions:}
        \begin{itemize}
            \item \textbf{Randomization}: Treatments randomly assigned within blocks
            \item \textbf{Replication}: Each treatment appears in each block
            \item \textbf{Independence}: Observations are independent given the design
            \item \textbf{Homoscedasticity }: Equal variances across treatments
            \item \textbf{Normality}: Residuals follow normal distribution
            %\item \textbf{Additivity}: No interaction between treatment and block effects
        \end{itemize}
    \end{block}

    \begin{alertblock}{Consequences of Assumption Violations:}
        \begin{itemize}
            \item \textbf{Invalid conclusions of parametric tests}: Need for non-parametric tests leading to reduced statistical power
        \end{itemize}
    \end{alertblock}
    
    \vfill
    {\tiny
    \begin{flushleft}
        Based on R. A. Fisher, Statistical Methods for Research Workers, in S. Kotz \& N. L. Johnson (eds.), Breakthroughs in Statistics: Methodology and Distribution, pp. 66--70, Springer, New York, 1992.
    \end{flushleft}
    }
\end{frame}

% Slide 4: Right Blocking
\begin{frame}
    \frametitle{The Right Blocking: Capturing Environmental Variability}
    
    \begin{columns}
        \begin{column}{0.7\textwidth}
            \begin{center}
                \begin{tikzpicture}[scale=0.8]
                    % Define colors for treatments
                    \definecolor{control}{RGB}{220,220,220}
                    \definecolor{tested}{RGB}{173,216,230}
                    \definecolor{reference}{RGB}{255,182,193}
                    
                    % Draw environmental variability strips (background)
                    \def\xoffset{0.1}
                    \def\yoffset{-0.15}
                    \fill[envlow] ({-0.5+\xoffset},{-0.5+\yoffset}) rectangle ({6.5+\xoffset},{1.5+\yoffset});
                    \fill[envmed] ({-0.5+\xoffset},{1.5+\yoffset}) rectangle ({6.5+\xoffset},{3+\yoffset});
                    \fill[envhigh] ({-0.5+\xoffset},{3+\yoffset}) rectangle ({6.5+\xoffset},{4.5+\yoffset});
                    
                    % Draw grid and plots
                    \foreach \row in {0,1,2} {
                        \foreach \col in {0,1,2} {
                            % Calculate position
                            \pgfmathsetmacro{\x}{\col*2}
                            \pgfmathsetmacro{\y}{\row*1.5}
                            
                            % Assign treatments (Latin square-like design)
                            \pgfmathsetmacro{\treatment}{int(mod(\col + \row, 3))}
                            \ifnum\treatment=0
                                \def\plotcolor{control}
                                \def\plotlabel{C}
                            \fi
                            \ifnum\treatment=1
                                \def\plotcolor{tested}
                                \def\plotlabel{T}
                            \fi
                            \ifnum\treatment=2
                                \def\plotcolor{reference}
                                \def\plotlabel{R}
                            \fi
                            
                            % Draw plot with some transparency to show environment
                            \fill[\plotcolor,opacity=0.8] (\x,\y) rectangle (\x+1.5,\y+1);
                            \draw[black, thick] (\x,\y) rectangle (\x+1.5,\y+1);
                            \node at (\x+0.75,\y+0.5) {\Large\textbf{\plotlabel}};
                        }
                        % Block labels
                        \node[left] at (-0.3,\row*1.5+0.5) {\textbf{Block \pgfmathprint{int(\row+1)}}};
                    }
                    
                    % Legend for treatments
                    \node[below] at (2,-1) {
                        \begin{tabular}{ll}
                            \textbf{C} & Control \\
                            \textbf{T} & Tested Product \\
                            \textbf{R} & Reference Product
                        \end{tabular}
                    };
                \end{tikzpicture}
            \end{center}
        \end{column}
        
        \begin{column}{0.3\textwidth}
            \begin{block}{\small Environmental Gradient:}
                \scriptsize
                \begin{tikzpicture}[scale=0.8]
                    % Color scale bar
                    \fill[envlow] (0,0) rectangle (1,0.5);
                    \fill[envmed] (0,0.5) rectangle (1,1);
                    \fill[envhigh] (0,1) rectangle (1,1.5);
                    \draw[black] (0,0) rectangle (1,1.5);
                    
                    % Labels
                    \node[right] at (1.1,0.25) {Low};
                    \node[right] at (1.1,0.75) {Medium};
                    \node[right] at (1.1,1.25) {High};
                    \node[below] at (0.5,-0.2) {\tiny Variability};
                \end{tikzpicture}
            \end{block}
        \end{column}
    \end{columns}
    
    %\vspace{0.5em}
    \begin{exampleblock}{\small Success of Blocking Strategy:}
        \begin{itemize}
            \scriptsize
            \item \textbf{Within-block homogeneity}: Treatments compared under similar conditions
            \item \textbf{Between-block heterogeneity}: Environmental gradient captured by block effects
            %\item \textbf{Improved precision}: Reduced experimental error through variance partitioning
        \end{itemize}
    \end{exampleblock}
\end{frame}

% Slide 5: The Wrong Blocking
\begin{frame}
    \frametitle{The Wrong Blocking: Assumption Violation}
    
    \begin{columns}
        \begin{column}{0.7\textwidth}
            \begin{center}
                \begin{tikzpicture}[scale=0.8]
                    % Define colors for treatments
                    \definecolor{control}{RGB}{220,220,220}
                    \definecolor{tested}{RGB}{173,216,230}
                    \definecolor{reference}{RGB}{255,182,193}
                    
                    % Draw irregular environmental variability shapes (background)
                    \def\xoffset{0.1}
                    \def\yoffset{-0.15}
                    
                    % Irregular environmental patches that don't align with blocks
                    % Low variability (diagonal patches)
                    \fill[envlow] ({0+\xoffset},{0+\yoffset}) -- ({2+\xoffset},{1+\yoffset}) -- ({3+\xoffset},{0.5+\yoffset}) -- ({4+\xoffset},{1.5+\yoffset}) -- ({6+\xoffset},{1+\yoffset}) -- ({6+\xoffset},{-0.5+\yoffset}) -- ({0+\xoffset},{-0.5+\yoffset}) -- cycle;
                    
                    % Medium variability (curved patches)
                    \fill[envmed] ({-0.5+\xoffset},{1.2+\yoffset}) -- ({1.5+\xoffset},{1.8+\yoffset}) -- ({3.5+\xoffset},{2.2+\yoffset}) -- ({5+\xoffset},{2.8+\yoffset}) -- ({6.5+\xoffset},{2.5+\yoffset}) -- ({6.5+\xoffset},{1+\yoffset}) -- ({4+\xoffset},{1.5+\yoffset}) -- ({3+\xoffset},{0.5+\yoffset}) -- ({2+\xoffset},{1+\yoffset}) -- ({0+\xoffset},{0+\yoffset}) -- ({-0.5+\xoffset},{0.8+\yoffset}) -- cycle;
                    
                    % High variability (irregular top patches)
                    \fill[envhigh] ({-0.5+\xoffset},{2.8+\yoffset}) -- ({2+\xoffset},{3.5+\yoffset}) -- ({4+\xoffset},{3.2+\yoffset}) -- ({6.5+\xoffset},{4.5+\yoffset}) -- ({6.5+\xoffset},{2.5+\yoffset}) -- ({5+\xoffset},{2.8+\yoffset}) -- ({3.5+\xoffset},{2.2+\yoffset}) -- ({1.5+\xoffset},{1.8+\yoffset}) -- ({-0.5+\xoffset},{1.2+\yoffset}) -- cycle;
                    
                    % Draw grid and plots
                    \foreach \row in {0,1,2} {
                        \foreach \col in {0,1,2} {
                            % Calculate position
                            \pgfmathsetmacro{\x}{\col*2}
                            \pgfmathsetmacro{\y}{\row*1.5}
                            
                            % Assign treatments (Latin square-like design)
                            \pgfmathsetmacro{\treatment}{int(mod(\col + \row, 3))}
                            \ifnum\treatment=0
                                \def\plotcolor{control}
                                \def\plotlabel{C}
                            \fi
                            \ifnum\treatment=1
                                \def\plotcolor{tested}
                                \def\plotlabel{T}
                            \fi
                            \ifnum\treatment=2
                                \def\plotcolor{reference}
                                \def\plotlabel{R}
                            \fi
                            
                            % Draw plot with some transparency to show environment
                            \fill[\plotcolor,opacity=0.8] (\x,\y) rectangle (\x+1.5,\y+1);
                            \draw[black, thick] (\x,\y) rectangle (\x+1.5,\y+1);
                            \node at (\x+0.75,\y+0.5) {\Large\textbf{\plotlabel}};
                        }
                        % Block labels
                        \node[left] at (-0.3,\row*1.5+0.5) {\textbf{Block \pgfmathprint{int(\row+1)}}};
                    }
                    
                    % Legend for treatments
                    \node[below] at (2,-1) {
                        \begin{tabular}{ll}
                            \textbf{C} & Control \\
                            \textbf{T} & Tested Product \\
                            \textbf{R} & Reference Product
                        \end{tabular}
                    };
                \end{tikzpicture}
            \end{center}
        \end{column}
        
        \begin{column}{0.3\textwidth}
            \begin{block}{\small Environmental Gradient:}
                \scriptsize
                \begin{tikzpicture}[scale=0.8]
                    % Color scale bar
                    \fill[envlow] (0,0) rectangle (1,0.5);
                    \fill[envmed] (0,0.5) rectangle (1,1);
                    \fill[envhigh] (0,1) rectangle (1,1.5);
                    \draw[black] (0,0) rectangle (1,1.5);
                    
                    % Labels
                    \node[right] at (1.1,0.25) {Low};
                    \node[right] at (1.1,0.75) {Medium};
                    \node[right] at (1.1,1.25) {High};
                    \node[below] at (0.5,-0.2) {\tiny Variability};
                \end{tikzpicture}
            \end{block}
        \end{column}
    \end{columns}
    
    \begin{alertblock}{\small Heteroscedasticity Assumption Violation Problem:}
        \scriptsize
        \begin{itemize}
            \item \textbf{Blocks fail to capture environmental variability}: Treatments compared under different conditions
            \item \textbf{Invalid parametric test}: Residual variance differs across treatments
        \end{itemize}
    \end{alertblock}
\end{frame}

% Slide 6: The Problem
\begin{frame}
    \frametitle{Current Limitations in Statistics for Agricultural Trials}
    
    \begin{block}{Traditional Approach Issues:}
        \begin{itemize}
            \item \textbf{Human-dependent blocking}: Environmental variability assessment relies on experimenter experience
            \item \textbf{A priori identification}: Must identify variance sources BEFORE data collection
        \end{itemize}
    \end{block}
    
    \begin{alertblock}{The Challenge:}
        \textit{How can we capture environmental variability mathematically rather than through human judgment?}
    \end{alertblock}
\end{frame}

% Slide 8: Research Gap
\begin{frame}
    \frametitle{The Missing Link: Spatial Coordinates}
    
    \begin{columns}
        \begin{column}{0.5\textwidth}
            \begin{block}{Geostatistical Methods Advantages:}
                \begin{itemize}
                    \item[\textcolor{green}{\checkmark}] \textbf{Mathematical modeling} of environmental variability
                    \item[\textcolor{green}{\checkmark}] \textbf{Post-hoc analysis} - no need for prior knowledge
                    \item[\textcolor{green}{\checkmark}] \textbf{Superior performance} in handling spatial heterogeneity
                    \item[\textcolor{green}{\checkmark}] \textbf{EPPO recognized} approach
                \end{itemize}
            \end{block}
        \end{column}
        
        \begin{column}{0.5\textwidth}
            \begin{alertblock}{Current Barrier:}
                \begin{itemize}
                    \item[\textcolor{red}{\times}] \textbf{Requires spatially referenced observations}
                    \item[\textcolor{red}{\times}] \textbf{Traditional manual assessments lack coordinates}
                    \item[\textcolor{red}{\times}] \textbf{Implementation gap} in practical field trials
                \end{itemize}
            \end{alertblock}
        \end{column}
    \end{columns}
\end{frame}

% Slide 7: Geostatistical Approach
\begin{frame}
    \frametitle{Geostatistical Approach: Spatial Linear Mixed Models}
    
    \begin{columns}
        \begin{column}{0.65\textwidth}
            \begin{center}
                \begin{tikzpicture}[scale=0.7]
                    % Define colors for treatments
                    \definecolor{control}{RGB}{220,220,220}
                    \definecolor{tested}{RGB}{173,216,230}
                    \definecolor{reference}{RGB}{255,182,193}
                    
                    % Draw irregular environmental variability shapes (background)
                    \def\xoffset{0.1}
                    \def\yoffset{-0.15}
                    
                    % Same irregular environmental patches as slide 5
                    \fill[envlow] ({0+\xoffset},{0+\yoffset}) -- ({2+\xoffset},{1+\yoffset}) -- ({3+\xoffset},{0.5+\yoffset}) -- ({4+\xoffset},{1.5+\yoffset}) -- ({6+\xoffset},{1+\yoffset}) -- ({6+\xoffset},{-0.5+\yoffset}) -- ({0+\xoffset},{-0.5+\yoffset}) -- cycle;
                    
                    \fill[envmed] ({-0.5+\xoffset},{1.2+\yoffset}) -- ({1.5+\xoffset},{1.8+\yoffset}) -- ({3.5+\xoffset},{2.2+\yoffset}) -- ({5+\xoffset},{2.8+\yoffset}) -- ({6.5+\xoffset},{2.5+\yoffset}) -- ({6.5+\xoffset},{1+\yoffset}) -- ({4+\xoffset},{1.5+\yoffset}) -- ({3+\xoffset},{0.5+\yoffset}) -- ({2+\xoffset},{1+\yoffset}) -- ({0+\xoffset},{0+\yoffset}) -- ({-0.5+\xoffset},{0.8+\yoffset}) -- cycle;
                    
                    \fill[envhigh] ({-0.5+\xoffset},{2.8+\yoffset}) -- ({2+\xoffset},{3.5+\yoffset}) -- ({4+\xoffset},{3.2+\yoffset}) -- ({6.5+\xoffset},{4.5+\yoffset}) -- ({6.5+\xoffset},{2.5+\yoffset}) -- ({5+\xoffset},{2.8+\yoffset}) -- ({3.5+\xoffset},{2.2+\yoffset}) -- ({1.5+\xoffset},{1.8+\yoffset}) -- ({-0.5+\xoffset},{1.2+\yoffset}) -- cycle;
                    
                    % Add treatment rectangles in center Y of plot
                    \pgfmathsetmacro{\midY}{1.8}
                    \fill[control,opacity=0.8] (0.5,\midY-0.3) rectangle (1.5,\midY+0.3);
                    \draw[black, thick] (0.5,\midY-0.3) rectangle (1.5,\midY+0.3);
                    \node at (1,\midY) {\textbf{C}};
                    
                    \fill[tested,opacity=0.8] (2.5,\midY-0.3) rectangle (3.5,\midY+0.3);
                    \draw[black, thick] (2.5,\midY-0.3) rectangle (3.5,\midY+0.3);
                    \node at (3,\midY) {\textbf{T}};
                    
                    \fill[reference,opacity=0.8] (4.5,\midY-0.3) rectangle (5.5,\midY+0.3);
                    \draw[black, thick] (4.5,\midY-0.3) rectangle (5.5,\midY+0.3);
                    \node at (5,\midY) {\textbf{R}};
                    
                    % Draw grid of observation points with coordinates
                    \foreach \row in {0,1,2,3,4} {
                        \foreach \col in {0,1,2,3,4,5} {
                            % Calculate position
                            \pgfmathsetmacro{\x}{\col*1.2}
                            \pgfmathsetmacro{\y}{\row*0.9}
                            
                            % Draw observation point
                            \fill[black] (\x,\y) circle (0.04);
                            
                            % Add coordinate labels for some points
                            \ifnum\row=0
                                \ifnum\col=0
                                    \node[below, tiny] at (\x,\y-0.1) {(0,0)};
                                \fi
                                \ifnum\col=3
                                    \node[below, tiny] at (\x,\y-0.1) {(3,0)};
                                \fi
                                \ifnum\col=5
                                    \node[below, tiny] at (\x,\y-0.1) {(5,0)};
                                \fi
                            \fi
                            \ifnum\col=0
                                \ifnum\row=2
                                    \node[left, tiny] at (\x-0.1,\y) {(0,2)};
                                \fi
                                \ifnum\row=4
                                    \node[left, tiny] at (\x-0.1,\y) {(0,4)};
                                \fi
                            \fi
                        }
                    }
                    
                    % Legend for treatments and spatial data
                    \node[below] at (3,-0.8) {
                        \tiny
                        \begin{tabular}{ll}
                            \textbf{C/T/R} & Control/Tested/Reference \\
                            \textbf{•} & Georeferenced observations
                        \end{tabular}
                    };
                \end{tikzpicture}
            \end{center}
        \end{column}
        
        \begin{column}{0.35\textwidth}
            \begin{block}{Spatial LMM:}
                \begin{equation*}
                    y(s_i) = \mu + \alpha_j + f(s_i) + \varepsilon_i
                \end{equation*}
                
                \small
                Where:
                \begin{itemize}
                    \item $y(s_i)$ = response at location $s_i$
                    \item $\mu$ = overall mean
                    \item $\alpha_j$ = treatment effect
                    \item $f(s_i)$ = spatial random field
                    \item $\varepsilon_i$ = measurement error
                    \item $s_i = (x_i, y_i)$ = coordinates
                \end{itemize}
            \end{block}
            
            \begin{exampleblock}{\small Key Advantage:}
                {\tiny Environmental variability is \textbf{modeled mathematically} through spatial coordinates rather than \textbf{assumed through blocking}}
            \end{exampleblock}
        \end{column}
    \end{columns}
    
    \begin{block}{\small Geostatistical Benefits:}
        \scriptsize
        \begin{itemize}
            \item \textbf{No blocking required}: Environmental variability captured by spatial correlation function
            \item \textbf{Post-hoc analysis}: No need to identify variance sources a priori
            \item \textbf{Optimal spatial interpolation}: Kriging provides best linear unbiased prediction
            \item \textbf{Uncertainty quantification}: Prediction variance maps available
        \end{itemize}
    \end{block}
\end{frame}

% Slide 9: Research Question
\begin{frame}
    \frametitle{Central Research Question}
    
    \begin{exampleblock}{}
        \large
        \textbf{Can geomatics technologies provide spatially referenced observations that enable geostatistical analysis within EPPO-compliant Plant Protection Product trials?}
    \end{exampleblock}
    
    \vspace{1em}
    
    \begin{block}{Specific Objectives:}
        \begin{enumerate}
            \item Establish minimum dataset requirements for digital data collection
            \item Demonstrate feasibility across all EPPO variable types
            \item Validate performance against traditional methods
            \item Provide practical implementation guidelines
        \end{enumerate}
    \end{block}
\end{frame}

% Slide 10: EPPO Standards Framework
\begin{frame}
    \frametitle{European Plant Protection Organization (EPPO)}
    
    \begin{block}{Key Standards:}
        \begin{itemize}
            \item \textbf{PP 1/152(4)}: Design and analysis of efficacy evaluation trials
            \item \textbf{PP 1/333(1)}: Digital technology adoption guidelines
        \end{itemize}
    \end{block}
    
    \begin{block}{Variable Types in EPPO Assessments:}
        \begin{enumerate}
            \item \textbf{Continuous/Discrete}: Plant counts, measurements
            \item \textbf{Ordinal}: Severity scales (0-100\%), damage ratings
            \item \textbf{Binary/Nominal}: Healthy/diseased, disease classification
        \end{enumerate}
    \end{block}
    
    \begin{alertblock}{Benchmark: R² > 0.85 compared to manual assessment}
    \end{alertblock}
\end{frame}

% Continue with more slides...
% For brevity, I'll show a few more key slides and provide the structure

% Slide 11: Plant Protection Products Context
\begin{frame}
    \frametitle{PPP Development \& Regulation}
    
    \begin{columns}
        \begin{column}{0.5\textwidth}
            \begin{block}{PPP Categories:}
                \begin{itemize}
                    \item Fungicides
                    \item Insecticides
                    \item Herbicides
                    \item Plant growth regulators
                    \item Acaricides
                    \item Nematicides
                \end{itemize}
            \end{block}
        \end{column}
        
        \begin{column}{0.5\textwidth}
            \begin{block}{Critical Evaluation Needs:}
                \begin{itemize}
                    \item \textbf{Efficacy}: Does it work?
                    \item \textbf{Selectivity}: Is it safe for crops?
                    \item \textbf{Environmental impact}: Side effects?
                \end{itemize}
            \end{block}
        \end{column}
    \end{columns}
\end{frame}

% Slide with image example
\begin{frame}
    \frametitle{Technical Arsenal}
    
    \begin{columns}
        \begin{column}{0.6\textwidth}
            \begin{block}{Core Technologies:}
                \begin{itemize}
                    \item \textbf{Photogrammetry}: 3D model generation from 2D images
                    \item \textbf{Spectral Imaging}: Multi/hyperspectral sensors
                    \item \textbf{Machine Learning}: Object detection, classification, regression
                    \item \textbf{GNSS/UAV}: Precise spatial positioning
                \end{itemize}
            \end{block}
        \end{column}
        
        \begin{column}{0.4\textwidth}
            % Example of how to include an image
            % \includegraphics[width=\textwidth]{images/technology_overview.png}
            \begin{center}
                \textit{[Technology diagram would go here]}
            \end{center}
        \end{column}
    \end{columns}
\end{frame}

% Study 1 Section
\section{Study 1: Plant Counting}

\begin{frame}
    \frametitle{Study 1: Automated Plant Counting}
    
    \begin{columns}
        \begin{column}{0.6\textwidth}
            \begin{block}{Problem Statement:}
                Manual plant counting is:
                \begin{itemize}
                    \item \textbf{Time-consuming}: Hours per plot
                    \item \textbf{Subjective}: Inter-observer variability
                    \item \textbf{Error-prone}: Missed or double-counted plants
                    \item \textbf{Non-spatial}: No coordinate information
                \end{itemize}
            \end{block}
            
            \begin{exampleblock}{Solution Approach:}
                \begin{itemize}
                    \item UAV photogrammetry
                    \item Deep learning object detection
                    \item Automatic spatial referencing
                    \item R² > 0.85 validation
                \end{itemize}
            \end{exampleblock}
        \end{column}
        
        \begin{column}{0.4\textwidth}
            % Image placeholder
            \begin{center}
                \textit{[Plant counting example image]}
            \end{center}
        \end{column}
    \end{columns}
\end{frame}

% Results slide with table
\begin{frame}
    \frametitle{Minimum Dataset Findings}
    
    \begin{table}
        \centering
        \begin{tabular}{|l|c|c|}
        \hline
        \textbf{Architecture} & \textbf{Images Needed} & \textbf{R²} \\
        \hline
        RT-DETR (Transformer-mixed) & 60 & 0.89 \\
        YOLOv8 (CNN) & 110 & 0.87 \\
        YOLOv5 (CNN) & 130 & 0.86 \\
        Few-shot models & N/A & < 0.85 \\
        Zero-shot models & N/A & < 0.85 \\
        \hline
        \end{tabular}
        \caption{Performance comparison across architectures}
    \end{table}
    
    \begin{alertblock}{Critical Finding:}
        \textbf{NO out-of-distribution trained model achieved R² > 0.85}\\
        \textit{In-domain training data is essential for agricultural applications}
    \end{alertblock}
\end{frame}

% Thank you slide
\begin{frame}
    \frametitle{Thank You}
    
    \begin{center}
        \Huge Questions \& Discussion
    \end{center}
    
    \vspace{2em}
    
    \begin{block}{Contact Information:}
        \textbf{Samuele Bumbaca}\\
        University of Turin\\
        Email: samuele.bumbaca@unito.it
    \end{block}
    
    \begin{block}{Key Publications:}
        \begin{enumerate}
            \item "On the Minimum Dataset Requirements..." - \textit{Remote Sensing} (2025)
            \item "Supporting Screening of New Plant Protection Products..." - \textit{Agronomy} (2024)
            \item "Anomaly Detection for Plant Disease Classification" - \textit{In preparation}
        \end{enumerate}
    \end{block}
\end{frame}

\end{document}