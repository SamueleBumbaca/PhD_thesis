\documentclass[11pt,a4paper]{article}
\usepackage[utf8]{inputenc}
\usepackage[T1]{fontenc}
\usepackage[margin=2.5cm]{geometry}
\usepackage{setspace}
\usepackage{titlesec}
\usepackage{fancyhdr}
\usepackage{hyperref}
\usepackage{enumitem}

% Font and spacing
\usepackage{lmodern}
\setstretch{1.2}

% Header and footer
\pagestyle{fancy}
\fancyhf{}
\fancyhead[L]{Response to Prof. Giovanna Sona}
\fancyhead[R]{S. Bumbaca - PhD Thesis Revision}
\fancyfoot[C]{\thepage}
\renewcommand{\headrulewidth}{0.4pt}
\renewcommand{\footrulewidth}{0pt}

% Title formatting
\titleformat{\section}{\large\bfseries}{\thesection}{1em}{}
\titleformat{\subsection}{\normalsize\bfseries}{\thesubsection}{1em}{}

\begin{document}

\begin{center}
{\LARGE\bfseries RESPONSE TO REVIEWER: PROFESSOR GIOVANNA SONA}\\[0.5cm]
{\large PhD Thesis Revision}\\[0.3cm]
{\normalsize Samuele Bumbaca}\\[0.2cm]
{\normalsize July 12, 2025}
\end{center}

\vspace{1cm}

Dear Professor Sona,

Thank you for your thorough review of my PhD thesis "Geomatic Techniques to Support Phytosanitary Products Tests within the EPPO Standard Framework" and for your valuable feedback. I appreciate your constructive comments and suggestions for improvement. Below is my detailed point-by-point response to each of your concerns, with specific references to the revisions made.

\section{Response to Structural and Organizational Issues}

\subsection{Response to: "The thesis needs a deep revision. It has a very confused structure..."}

\textbf{REVISION MADE:} I have completely restructured the thesis to provide a clearer narrative flow and better highlight the research contributions. The new structure includes:

\begin{itemize}
\item \textbf{Added Extended Abstract} (pages 2-3 in revised version vs. absent in original): A comprehensive abstract now provides a clear overview of the research background, objectives, methodology, key results, novel contributions, and practical impact.

\item \textbf{Enhanced Introduction Chapter} (pages 4-8 in revised version vs. pages 2-4 in original): The introduction now includes three well-defined sections:
\begin{itemize}
\item Research Overview and Motivation (clearly establishing the problem)
\item Research Objectives (specific aims and variable types addressed)
\item Thesis Structure (roadmap for the reader)
\end{itemize}

\item \textbf{Added Literature Review Section} (pages 8-10 in revised version vs. absent in original): A dedicated section now covers current limitations, geostatistical approaches, digital technologies, and clearly identifies the research gap.
\end{itemize}

\subsection{Response to: "The thesis is missing a real introductory chapter..."}

\textbf{REVISION MADE:} The thesis now begins with a comprehensive abstract (pages 2-3) that serves as the requested summary, followed by a restructured Introduction chapter that clearly presents:

\begin{itemize}
\item \textbf{Research Topic}: Integration of geomatics technologies with geostatistical methods for PPP evaluation
\item \textbf{Purpose}: Enable adoption of geostatistical methods by providing georeferenced observations
\item \textbf{Workflow}: Three case studies addressing continuous/discrete, ordinal, and binary/nominal variables
\end{itemize}

\subsection{Response to: "Then should follow a state of the art and literature review part..."}

\textbf{REVISION MADE:} Added a dedicated "Literature Review and Research Gap" section (pages 8-10) that includes:
\begin{itemize}
\item Current limitations in agricultural statistical design
\item Geostatistical approaches in agricultural research
\item Digital technologies in agricultural assessment
\item Clear identification of the research gap and innovation
\end{itemize}

\section{Response to Chapter Organization Issues}

\subsection{Response to: "Chapter 1 indeed is not an introduction, but a list of methods..."}

\textbf{REVISION MADE:} The former Chapter 1 has been restructured as Chapter 2 "Theoretical Background and Methodology" (pages 10-47), which now properly contextualizes the methods within the research framework rather than presenting them as isolated techniques.

\subsection{Response to: "Paragraph 1.4 must be moved at the beginning"}

\textbf{REVISION MADE:} The content that was in paragraph 1.4 ("The Literature Gap and the Thesis Aims") has been integrated into the new Introduction chapter structure, with research objectives now appearing in section 1.2 (pages 5-6) and the literature gap discussion in section 1.4 (pages 8-10).

\subsection{Response to: "Many different statistical methods are mentioned... confusing"}

\textbf{REVISION MADE:} Statistical methods are now organized logically within the "Theoretical Background and Methodology" chapter, with clear subsections for:
\begin{itemize}
\item Regulatory Framework (Section 2.1)
\item EPPO Standards (Section 2.2)
\item Geomatics techniques (Section 2.3)
\end{itemize}

Each method is introduced with its specific purpose and application context.

\section{Response to Technical and Methodological Issues}

\subsection{Response to: "Ch.1.3 could be split into geostatistic part and digital images part..."}

\textbf{REVISION MADE:} The geomatics section (now Section 2.3, pages 21-47) has been reorganized into clear subsections:
\begin{itemize}
\item Photogrammetry and 3D reconstruction
\item Spectral imaging and feature extraction
\item Machine learning applications
\item Geostatistical methods
\end{itemize}

The photogrammetric workflow and its products (DEM, orthophotos) are now clearly described before introducing how these enable index calculation and further analysis.

\subsection{Response to: "Vague definitions should be replaced with more appropriate ones..."}

\textbf{REVISION MADE:} All technical definitions have been revised for precision and accuracy. For example, photogrammetry is now defined more precisely in the context of 3D reconstruction and spatial measurement, with specific reference to the mathematical principles involved.

\subsection{Response to: "Substantial mistake in photogrammetric workflow description..."}

\textbf{REVISION MADE:} The photogrammetric workflow description has been corrected to accurately reflect that exterior orientation requires ground control points (GCPs) or other georeferencing methods in addition to tie points. The revised text now properly distinguishes between relative orientation (using tie points) and absolute orientation (requiring GCPs or other georeferencing).

\section{Response to Documentation and Reference Issues}

\subsection{Response to: "Published parts should be mentioned..."}

\textbf{REVISION MADE:} All published sections are now clearly identified with full citation information:
\begin{itemize}
\item Plant counting case study: "On the Minimum Dataset Requirements for Fine-Tuning an Object Detector for Arable Crop Plant Counting" (MDPI Remote Sensing, DOI: 10.3390/rs17132190)
\item Phytotoxicity scoring case study: "Supporting Screening of New Plant Protection Products through a Multispectral Photogrammetric Approach Integrated with AI" (MDPI Agronomy, DOI: 10.3390/agronomy14020306)
\end{itemize}

\subsection{Response to: "Reference lists are in weird format..."}

\textbf{REVISION MADE:}
\begin{itemize}
\item Bibliography format has been standardized to numbered citations with square brackets using the natbib package
\item All bibliographic entries have been consolidated into a single merged bibliography file
\item Missing references have been completed with full citation information
\item Bibliography style changed from mixed format to consistent "unsrt" style
\end{itemize}

\section{Response to Language and Style Issues}

\subsection{Response to: "Many typos have been found, starting from the title itself..."}

\textbf{REVISION MADE:}
\begin{itemize}
\item The title has been corrected from "whithin" to "within"
\item Comprehensive English language revision has been performed throughout the document
\item Technical terminology has been standardized
\item Grammar and syntax have been improved throughout
\end{itemize}

\section{Summary of Major Structural Changes}

\subsection{Original Version Structure:}
\begin{itemize}
\item Chapter 1: Introduction (mixed content)
\item Chapter 2: Study Cases (brief)
\item Chapter 3: Conclusions
\end{itemize}

\subsection{Revised Version Structure:}
\begin{itemize}
\item Extended Abstract (NEW)
\item Chapter 1: Introduction (completely restructured)
\begin{itemize}
\item Research Overview and Motivation
\item Research Objectives
\item Thesis Structure
\item Literature Review and Research Gap
\end{itemize}
\item Chapter 2: Theoretical Background and Methodology (reorganized)
\item Chapter 3: Applications Demonstration - Case Studies (expanded)
\item Chapter 4: Conclusions (enhanced)
\end{itemize}

\section{Conclusion}

The revised thesis now provides a clear research narrative that addresses all your concerns about structure, clarity, and technical accuracy. The reorganization makes the research contributions and their significance much more apparent to readers.

Thank you again for your thorough review. I believe these revisions have significantly improved the thesis and addressed all the major concerns you raised.

\vspace{1cm}

Sincerely,\\
Samuele Bumbaca

\vspace{0.5cm}

Date: July 12, 2025

\end{document}
