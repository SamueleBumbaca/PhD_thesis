\documentclass[11pt,a4paper]{article}
\usepackage[utf8]{inputenc}
\usepackage[T1]{fontenc}
\usepackage[margin=2.5cm]{geometry}
\usepackage{setspace}
\usepackage{titlesec}
\usepackage{fancyhdr}
\usepackage{hyperref}
\usepackage{enumitem}

% Font and spacing
\usepackage{lmodern}
\setstretch{1.2}

% Header and footer
\pagestyle{fancy}
\fancyhf{}
\fancyhead[L]{Response to Prof. Daniela Stroppiana}
\fancyhead[R]{S. Bumbaca - PhD Thesis Revision}
\fancyfoot[C]{\thepage}
\renewcommand{\headrulewidth}{0.4pt}
\renewcommand{\footrulewidth}{0pt}

% Title formatting
\titleformat{\section}{\large\bfseries}{\thesection}{1em}{}
\titleformat{\subsection}{\normalsize\bfseries}{\thesubsection}{1em}{}

\begin{document}

\begin{center}
{\LARGE\bfseries RESPONSE TO REVIEWER: PROFESSOR DANIELA STROPPIANA}\\[0.5cm]
{\large PhD Thesis Revision}\\[0.3cm]
{\normalsize Samuele Bumbaca}\\[0.2cm]
{\normalsize July 12, 2025}
\end{center}

\vspace{1cm}

Dear Dr. Stroppiana,

Thank you for your detailed and constructive review of my PhD thesis "Geomatic Techniques to Support Phytosanitary Products Tests within the EPPO Standard Framework." I greatly appreciate your thorough evaluation and valuable suggestions for improvement. Below is my detailed point-by-point response to each of your comments, with specific references to the revisions made.

\section{Response to Overall Structure and Organization}

\subsection{Response to: "The thesis document... is not well structured and organized..."}

\textbf{REVISION MADE:} I have completely restructured the thesis to provide a coherent and logical flow. The new structure includes:

\begin{itemize}
\item \textbf{Extended Abstract} (pages 2-3 in revised version vs. absent in original): Provides comprehensive overview of background, objectives, methodology, results, and contributions
\item \textbf{Reorganized Introduction} (pages 4-10 in revised version vs. pages 2-4 in original): Now includes clear research motivation, objectives, structure, and literature review
\item \textbf{Enhanced Theoretical Background} (pages 10-47 in revised version): Better organized with logical progression from regulatory framework to technical methods
\item \textbf{Improved Case Studies Section} (pages 47-52 in revised version): Clearer presentation of the three applications with better contextualization
\end{itemize}

\subsection{Response to: "Major weakness... lacks overall view of scientific work..."}

\textbf{REVISION MADE:} The thesis now prominently features the novel contributions:

\begin{itemize}
\item \textbf{In the Abstract} (pages 2-3): "Novel Contribution" subsection explicitly states: "This work establishes the first systematic evaluation of minimum requirements for implementing geomatics techniques within EPPO standards, providing practical guidelines for dataset size, validation protocols, and integration strategies."

\item \textbf{In the Conclusions} (pages 52-55): Added "Geomatic Contributions and Innovations" section that clearly articulates three critical advantages provided by geomatics techniques: spatial data integration, enhanced data density, and reproducibility/standardization.
\end{itemize}

\section{Response to Results and Summary Issues}

\subsection{Response to: "Better highlight results from three case studies..."}

\textbf{REVISION MADE:}

\begin{itemize}
\item \textbf{In the Abstract} (pages 2-3): "Key Results" subsection now provides quantitative summary of all three case studies with specific performance metrics
\item \textbf{In the Applications Demonstration chapter} (pages 47-52): Each case study now has clearer results presentation with EPPO benchmark achievements highlighted
\item \textbf{In the Conclusions} (pages 52-55): Comprehensive summary of findings from all three case studies with their implications for the field
\end{itemize}

\subsection{Response to: "The text is verbose... should be shortened..."}

\textbf{REVISION MADE:} The thesis has been significantly condensed while maintaining scientific rigor:
\begin{itemize}
\item Removed redundant background descriptions
\item Streamlined methodological sections to focus on novel contributions
\item Consolidated repetitive content
\item Improved conciseness throughout while preserving technical accuracy
\end{itemize}

\section{Response to Abstract and Introduction Issues}

\subsection{Response to: "Extended abstract is missing..."}

\textbf{REVISION MADE:} Added comprehensive extended abstract (pages 2-3) that includes:
\begin{itemize}
\item \textbf{Background and Research Gap}: Clear statement of the problem and limitations of current approaches
\item \textbf{Research Objectives}: Specific aims with reference to EPPO standards
\item \textbf{Methodology}: Overview of the three-pronged approach (counting, scoring, classification)
\item \textbf{Key Results}: Quantitative results with specific performance metrics
\item \textbf{Novel Contribution}: Explicit statement of the first systematic evaluation of geomatics requirements
\item \textbf{Practical Impact}: Clear implications for agricultural research and regulatory frameworks
\end{itemize}

\subsection{Response to: "Clearly describe contribution to methodology and results..."}

\textbf{REVISION MADE:} Personal contributions are now clearly highlighted throughout:
\begin{itemize}
\item \textbf{In the Abstract}: "Novel Contribution" section explicitly states original research contributions
\item \textbf{In the Introduction}: Research objectives clearly delineate the candidate's specific investigations
\item \textbf{In the Methodology}: Distinguishes between established techniques and novel applications/combinations
\item \textbf{In the Results}: Emphasizes original findings and their significance for the field
\item \textbf{Language}: Changed from first person plural to appropriate academic voice while maintaining clarity about personal contributions
\end{itemize}

\section{Response to Content Balance and Focus Issues}

\subsection{Response to: "Most of the text describes theoretical background..."}

\textbf{REVISION MADE:} Rebalanced the content to emphasize original research:
\begin{itemize}
\item \textbf{Theoretical Background chapter}: Streamlined to focus on methods relevant to the thesis applications
\item \textbf{Introduction}: Now clearly connects background to specific research gaps addressed
\item \textbf{Literature Review}: Focused on identifying gaps that the thesis addresses rather than general background
\item \textbf{Results sections}: Expanded to better highlight novel findings and their implications
\end{itemize}

\subsection{Response to: "Geomatic techniques focus should be first..."}

\textbf{REVISION MADE:} The Geomatics section (now Section 2.3, pages 21-47) has been reorganized with:
\begin{itemize}
\item \textbf{Clear definition upfront}: Geomatic techniques are introduced immediately with their relevance to the research
\item \textbf{Logical progression}: From basic principles to specific applications in agricultural assessment
\item \textbf{Focus on thesis applications}: Each technique is presented in the context of how it addresses the research objectives
\end{itemize}

\section{Response to Objectives and Case Studies Issues}

\subsection{Response to: "Objectives need separate section at beginning..."}

\textbf{REVISION MADE:} Added dedicated "Research Objectives" section (pages 5-6) that clearly states:
\begin{itemize}
\item The overarching goal of investigating geomatics applicability for EPPO standards
\item The three specific variable types addressed (continuous/discrete, ordinal, binary/nominal)
\item The representative assessments selected for each type
\item The feasibility demonstration goals
\end{itemize}

\subsection{Response to: "More specific on study cases from early beginning..."}

\textbf{REVISION MADE:} Study cases are now clearly identified from the beginning:
\begin{itemize}
\item \textbf{In the Abstract}: Specific mention of "counting, scoring, and classification" applications
\item \textbf{In the Introduction}: Clear enumeration of the three case studies with their specific applications
\item \textbf{In the Thesis Structure section}: Explicit listing of each case study with publication status
\item \textbf{Eliminated vague references}: No more ambiguous terms like "series of study cases" or "second study case"
\end{itemize}

\section{Response to Methodology and Variable Type Issues}

\subsection{Response to: "Methodology lost in details..."}

\textbf{REVISION MADE:} Methodology sections have been streamlined:
\begin{itemize}
\item \textbf{Focus on novel aspects}: Emphasis on original contributions rather than standard procedures
\item \textbf{Clearer organization}: Logical flow from general principles to specific applications
\item \textbf{Reduced redundancy}: Eliminated repetitive descriptions of standard techniques
\item \textbf{Enhanced clarity}: Better integration of methodological choices with research objectives
\end{itemize}

\subsection{Response to: "Binary, ordinal concepts are very general..."}

\textbf{REVISION MADE:} Variable types are now clearly connected to specific applications:
\begin{itemize}
\item \textbf{Continuous/Discrete Variables}: Explicitly identified as plant counting applications
\item \textbf{Ordinal Variables}: Clearly stated as phytotoxicity scoring
\item \textbf{Binary/Nominal Variables}: Specified as plant disease detection/classification
\item \textbf{Throughout the text}: Consistent use of specific variable names rather than general categories
\end{itemize}

\section{Response to Conclusions and Future Work Issues}

\subsection{Response to: "Conclusions need to discuss geomatic contributions..."}

\textbf{REVISION MADE:} Enhanced Conclusions section (pages 52-55) now includes:
\begin{itemize}
\item \textbf{Dedicated "Geomatic Contributions and Innovations" section}: Three critical advantages provided by geomatics techniques
\item \textbf{Balanced discussion}: Equal emphasis on geomatics and ML contributions
\item \textbf{Clear connection to thesis title}: Emphasis on how geomatics enables spatial data integration for EPPO standards
\item \textbf{Specific geomatics benefits}: Spatial data integration, enhanced data density, reproducibility/standardization
\end{itemize}

\subsection{Response to: "Add ideas on future developments..."}

\textbf{REVISION MADE:} Added "Future Research Directions" section (pages 54-55) that includes:
\begin{itemize}
\item \textbf{Temporal Geostatistics}: Extending spatial modeling to include temporal dimensions
\item \textbf{Multi-sensor Fusion}: Combining thermal, LiDAR, and hyperspectral data
\item \textbf{Real-time Processing}: Edge computing solutions for in-field analysis
\item \textbf{Regulatory Integration}: Collaboration with EPPO for standardized digital protocols
\end{itemize}

\section{Response to Language and Style Issues}

\subsection{Response to: "Misspelling in title words"}

\textbf{REVISION MADE:} Title corrected from "whithin" to "within" throughout the document.

\subsection{Response to: "Rephrase too general sentences..."}

\textbf{REVISION MADE:} All identified general statements have been revised:
\begin{itemize}
\item Replaced vague terms like "supposedly nadiral" with precise technical language
\item Clarified "handcrafted algorithm" references with specific algorithmic approaches
\item Eliminated overly broad generalizations about automated systems
\item Enhanced technical precision throughout the document
\end{itemize}

\subsection{Response to: "Clear description of validation..."}

\textbf{REVISION MADE:} Validation terminology has been clarified:
\begin{itemize}
\item \textbf{Clear definitions}: Distinguished between testing, validation, cross-validation, and independent validation
\item \textbf{Specific benchmarks}: EPPO validation standards are explicitly referenced with their sources
\item \textbf{Methodological clarity}: Validation approaches are described with specific reference to their application context
\end{itemize}

\subsection{Response to: "Avoid first plural pronoun..."}

\textbf{REVISION MADE:} Revised throughout to use appropriate academic voice:
\begin{itemize}
\item Eliminated first person plural pronouns
\item Used passive voice where appropriate
\item Maintained clarity while emphasizing the personal nature of the doctoral research
\item Balanced between personal contribution acknowledgment and professional academic tone
\end{itemize}

\section{Summary of Major Improvements}

\subsection{Structural Enhancements:}
\begin{itemize}
\item Added comprehensive extended abstract
\item Reorganized Introduction with clear research motivation and objectives
\item Enhanced literature review with focused gap identification
\item Improved case study presentation with clearer results
\end{itemize}

\subsection{Content Improvements:}
\begin{itemize}
\item Emphasized novel geomatics contributions alongside ML applications
\item Clarified specific applications for each variable type
\item Enhanced conclusions with future research directions
\item Corrected technical inaccuracies and improved precision
\end{itemize}

\subsection{Language and Style:}
\begin{itemize}
\item Corrected title spelling error
\item Improved academic voice and tone
\item Enhanced clarity and conciseness
\item Eliminated vague generalizations
\end{itemize}

\section{Conclusion}

The revised thesis now provides a much clearer presentation of the research contributions, with better organization, enhanced focus on geomatics innovations, and clearer connection between methodology and results. The document should now be much more accessible and informative for readers while maintaining scientific rigor.

Thank you again for your thorough and constructive review. These revisions have significantly strengthened the thesis and better highlight the contributions of geomatics techniques to agricultural research.

\vspace{1cm}

Sincerely,\\
Samuele Bumbaca

\vspace{0.5cm}

Date: July 12, 2025

\end{document}
