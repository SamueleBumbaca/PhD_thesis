\documentclass[12pt,a4paper,oneside]{report}

% Page layout
\usepackage[a4paper,top=1.7cm,bottom=7.4cm,left=2.5cm,right=6.0cm,footskip=6.3cm]{geometry}
\usepackage{setspace}
\usepackage{titlesec}
\usepackage{titling}
\usepackage{times}
\usepackage{fontspec}
\setmainfont{Arial}
\usepackage{fancyhdr}
\pagestyle{fancy}
\fancyhf{}
\fancyfoot[C]{\thepage}
\renewcommand{\headrulewidth}{0pt}
\renewcommand{\footrulewidth}{0pt}

% Include PDF
\usepackage{pdfpages}
% Bibliography
\usepackage{natbib}
\bibliographystyle{plain}  % or another style that suits your needs
\usepackage{url}
\usepackage{hyperref}

% Paragraph formatting
\setlength{\parskip}{6pt}
\setlength{\parindent}{0pt}
\setstretch{1.5}

\begin{document}

% Include the first page from the PDF file
\includepdf[pages=1]{Intestazione/t3._Thesis_first_page.pdf}

% Table of Contents
\tableofcontents
\newpage

% Main content
\chapter{Introduction}

\section{EPPO}
\subsection{Phytosanitary Products}

Phytosanitary products, commonly used as a synonym for "Plant Protection Products" (PPPs),
are a specific category of pesticides designed primarily to maintain crop health 
and prevent destruction by diseases and infestations. While the term "pesticides" 
is broader and also includes biocidal products used to control harmful organisms 
and disease carriers not related to plant protection, phytosanitary products are 
specifically used to control harmful organisms affecting cultivated plants (such 
as insects, mites, fungi, bacteria, rodents, etc.), eliminate weeds, and regulate 
plant physiological processes. Fertilizers, which serve for plant nutrition and 
soil fertility improvement, are excluded from phytosanitary products.

Phytosanitary products contain at least one active substance, which can be either 
chemical compounds or microorganisms, including viruses, that enable the product 
to perform its intended function. These active substances undergo rigorous risk 
assessment processes, with EFSA (European Food Safety Authority) playing a central 
role in conducting peer reviews at the EU level to determine if these products, 
when used correctly, might produce harmful effects on human or animal health, either 
directly or indirectly through drinking water, food, or feed.

The main categories of phytosanitary products can be distinguished based on the 
type of organism they target or the function they perform, including:


\begin{itemize}
    \item Fungicides
    \item Insecticides
    \item Acaricides
    \item Rodenticides
    \item Slimicides
    \item Nematicides
    \item Herbicides
    \item Plant growth regulators
\end{itemize}

The parameters identified through the risk assessment are compared with the values 
established by directive 97/57/EC \cite{EURLex1997265}, which indicates the acceptability limits for 
decision-making on the inclusion of active substances in the EU list (Annex I of 
directive 91/414/EEC \cite{directive_91_414_EEC}).

The Introduction of a product in the EU market is not only subject to audits on
active substances and their safety for humans and environment but also to the evaluation 
of the product's efficacy and safety for the crop.
World Trade Organization Sanitary and Phytosanitary Measures Agreement \cite{WTO_SPS_Agreement}
recognizes the International Plant Protection Convention (IPPC) as the only international
institution in charge of emitting standards for plant health \cite{IPPC}. IPPC is organized in
regions. European Union (EU) countries refer to the European and Mediterranean Plant
Protection Organization (EPPO). EPPO Standards are divided into Standards on
Phytosanitary Measures and Standards on PPPs. PPPs standards describe the efficacy
evaluation of PPPs (PP 1) and good plant protection practices. EU GEP units provide
Biological Assessment Dossier (BAD) efficacy trials. GEP units are expected to follow
EPPO PP 1 to assess PPPs selectivity detecting phytotoxicity effects, and efficacy in the
complaint of Regulation (EC) No 1107/2009 of the European Parliament and Council \cite{EC_Regulation_1107_2009}.

\subsection{Standard Experimental Design}

Generics on efficacy assessments are reported in PP 1/181(5) \cite{EPPO_PP1_181}, which describes
herbicide, fungicide, bactericide, and insecticide efficacy on the target evaluation.
PP 1/135(4) \cite{EPPO_PP1_135} describes the selectivity assessment procedures, 
in other words: the standard phytotoxicity assessments of PPPs.
The PP 1/152 \cite{EPPO_PP1_152} standard describes the general principles for the
efficacy and selectivity evaluation of PPPs, in describing the standard experimental design.
Aside from the objectives of the study and the description of thesis (treatments), 
% controls, reference treatment, plot size, replications, randomization, sampling and assessment timing, 
the PP 1/152 outlined that a comprehensive experimental design should include a description of:
\begin{itemize}
    \item \textbf{Type of Design}
    \item \textbf{Sampling Method and Measures Units}
    \item \textbf{Statistical Analysis Plan}
\end{itemize}

For what concerns the type of design, EPPO "envisage trials in which the experimental
treatments are the ‘test product(s), reference product(s) and
untreated control, arranged in a suitable statistical design’" \cite{EPPO_PP1_152}.
The experimental design should be randomized, with replications and blocks, and
should include a sufficient number of plots to ensure the statistical power of the
analysis. The number of replications and blocks should be determined based on the
expected variability of the data and the desired level of statistical significance
in respect control and reference thesis. The
randomization of thesis within blocks should be carried out using a suitable
randomization procedure to ensure that the treatments are assigned to plots in a
completely random manner. The key randomization used in phytosanitary product 
evaluations include:

\begin{itemize}
    \item \textbf{Completely Randomized Design (CRD)}: Treatments randomly assigned to 
          experimental units; statistically powerful but only suitable for homogeneous trial 
          areas where environmental variation is minimal.
          
    \item \textbf{Randomized Complete Block Design (RCBD)}: Groups plots into homogeneous 
          blocks with each treatment appearing once per block; controls for environmental 
          heterogeneity across the experimental area.
          
    \item \textbf{Split-Plot Design}: Used when one factor (e.g., cultivation equipment) 
          cannot be fully randomized; creates hierarchy with whole plots and subplots; 
          particularly useful when plot size or equipment constraints exist.
          
    \item \textbf{Systematic designs}: Non-randomized arrangements rarely suitable for 
          efficacy evaluations; may only be appropriate in special cases like varietal 
          trials on herbicide selectivity.
\end{itemize}

When designing phytosanitary product trials, the arrangement of untreated controls 
is critical for proper efficacy assessment. According to EPPO standards, the main 
purpose of untreated controls is to demonstrate adequate pest infestation, without 
which efficacy cannot be meaningfully evaluated. Four distinct arrangements for 
untreated controls exist:

\begin{itemize}
    \item \textbf{Included controls}: The most common approach, where control plots 
    have the same shape and size as treatment plots and are fully randomized within 
    the experimental design. This arrangement is essential when controls 
    will be used in statistical comparisons.
    
    \item \textbf{Imbricated controls}: Control plots are arranged systematically 
    within the trial (between blocks or between treated plots), potentially with 
    different dimensions than treatment plots. These observations are typically 
    not included in statistical analyses but ensure more homogeneous distribution 
    of untreated area effects.
    
    \item \textbf{Excluded controls}: Control plots are established outside the 
    main trial area but in similar environmental conditions. While replication is 
    not essential, it may be beneficial in heterogeneous environments. These observations 
    are generally excluded from statistical analyses.
    
    \item \textbf{Adjacent controls}: Each plot is divided into two subplots, with 
    one randomly selected to remain untreated. This approach is particularly valuable 
    in highly heterogeneous environments but requires specialized split-plot statistical 
    analysis.
\end{itemize}

The selection of control arrangement depends on several factors: whether the control 
will be included in statistical tests (requiring included controls), the degree 
of environmental heterogeneity (adjacent controls are preferred for high heterogeneity), 
and the potential for control plots to interfere with adjacent treatment plots (suggesting 
excluded controls when interference is likely).
The trials type design is critical for the success of the study, as it ensures
that the results are reliable, reproducible, and statistically valid.

After defining the experimental units through the randomization design choise,
the next step is to define the sampling method and the measures units.
Target and crop-specific standards point out "mode of
assessment recording and measurements" fixing evaluation metrics in two ways:
countable (discrete values) and measurable (continuous values) effects which must be
expressed in absolute values, in other cases, frequency (incidence) and degree
(severity) should be estimated and reported as affected percentage of the individual (ex.
plant or plot) or as proportion within thesis and control expressed in percentage. As
specified by PP 1/152 \cite{EPPO_PP1_152}, classification by ranking (ordinal) and scoring (ordinal or
nominal) is also contemplated. In the case of estimation, rather than count or measure,
PP 1/152 reports "The observer should be trained to make the estimations and his
observations should be calibrated against a standard". Calibration compliance with
standards is ensured by GEP audits. Scoring and ranking scales examples are
published on specific standards or the same PP 1/152. The lack of specific scales lets
trial protocol authors define one inspired in range and intervals by the mentioned
examples or other well-established ones.
GEP units PP 1 assessments are produced by trained and experienced
agronomists or biologists by visual inspection or laboratory analysis. The technician
follows the trial protocol and related EPPO standards during assessment execution. The
technician is critical for accuracy, precision, and repeatability. Sensitivity is determined
by the trial protocol. It depends on expected differences and if a measure, a proportion,
or a scale is used. For instance, in PP 1/93(3) \cite{EPPO_PP1_93} "Efficacy evaluation of herbicides -
Weeds in cereals - Observation on the crop", phytotoxicity color modification could be
measured, or estimated as proportion in respect to the untreated, or scored in EPPO
scale as PP 1/135(4) reports, or a scientifically accepted score as the European Weed
Research Society phytotoxicity damage score \cite{EWRS_score} and other ones.
In general, data types must undergo the classification presented in Table 1.1

\begin{table}[ht]
\caption{Different modes of observation and types of variables}
\label{tab:data_types}
\centering
\begin{tabular}{|l|c|c|c|c|}
\hline
\textbf{Type of Variable} & \textbf{Measurement} & \textbf{Visual Estimation} & \textbf{Ranking} & \textbf{Scoring} \\
\hline
Binary & & & & X \\
\hline
Nominal & & & & X \\
\hline
Ordinal & & & X & X \\
\hline
Discrete & X & X & & \\
\hline
Continuous limited & X & X & & \\
\hline
Continuous not limited & X & X & & \\
\hline
\end{tabular}
\end{table}

The statistical analysis of trials is equally critical, providing objective 
assessment of treatment effects. While PP 1/152 \cite{EPPO_PP1_152} doesn't prescribe 
specific analyses for all situations, it emphasizes that analysis methods should 
align with the experimental design and data types collected. For qunatitative 
variables (continuous or discrete), parametric methods based on Generalized 
Linear Models (GLM) are recommended, including ANOVA and regression approaches. 
For qualitative variables (ordinal or nominal), non-parametric methods are more 
appropriate. Parametric analysis assumes additivity of effects, homogeneity of variance, 
and normally distributed errors—when these assumptions aren't met, data transformations 
or alternative approaches become necessary.

Statistical tests, particularly F-tests of orthogonal 
contrasts, should focus on biologically relevant comparisons specified during the 
design stage: untreated control versus treatments (establishing trial validity), 
reference products versus control (demonstrating coherence), test products versus 
reference (evaluating efficacy), and comparisons among test products (identifying 
superior treatments). For efficacy trials, EPPO suggests one-sided tests since the 
aim is comparing products against references or controls, with appropriate multiple 
comparison procedures when needed.

Through adherence to these rigorous design and analysis standards, researchers can 
generate reliable evidence to support phytosanitary product registration while ensuring 
that products demonstrate consistent efficacy across relevant agricultural conditions.

\subsection{Digital Approaches}

While the EPPO experimental design standards provide a solid foundation for conducting
phytosanitary product trials, the increasing availability of digital tools and technologies
offers new opportunities to enhance the quality (in the "Quality of a mode of observation" sense \cite{EPPO_PP1_152}) 
and efficiency of these assessments.
Digital approaches can automate data collection and analysis, improving the
reproducibility of results, ultimately accelerating the development and registration of
effective phytosanitary products.

To regulate the use of this kind of technologies, the EPPO published a new standard, 
PP 1/333(1) \cite{PP13332024}, which 
filled the gap in the use of digital technologies in phytosanitary product efficacy
and selectivity trials. This standard provides guidelines for incorporating digital
tools into trial protocols, where digital tools are intended as a combination of
hardwares and softwares delivering sample measurments 
in a semi-automatic or automatic fashon.
The delivered measurments must respect the same quality standards of the manual
ones, and the digital tools must be validated before the trial execution.
The standard also provides guidance on the validation of digital tools, which should
be performed by comparing the results of digital and manual assessments. The validation
report should demonstrate that the digital tools provide reliable and consistent
results compared to manual assessments. The benchmarks for the validation depends
on the type of variables measured. For each type of variable, the congruence between 
digital and manual measurement should be measured with a different metric:
\begin{itemize}
    \item \textbf{Continuous}: Coefficient of determination (R²) higher than 0.85.
    \item \textbf{Ordinal and Nominal}: Cohen K higher than 0.7.
    \item \textbf{Binary}: Accuracy higher than 0.85
\end{itemize}

One of the most compelling advantages of adopting digital approaches is the
dramatic increase in the number of observations that can be collected. Unlike
manual methods, which are inherently limited by human capacity and time
constraints, automated and semi-automated systems can continuously gather data
with minimal interruption. This greater volume of data not only improves the
resolution and granularity of analysis but also significantly enhances the
statistical power of hypothesis testing.

The power of a statistical test, defined as the probability of correctly
rejecting the null hypothesis when it is false, directly depends on the sample
size (number of observations). As noted by classical statisticians such as
Student (1908), Pearson (1920), and Fisher (1935), increasing the independent sample size
is a fundamental way to achieve higher statistical power. However effiecient technics
to collect data does not garantee indipendency of samples. Wheter repeated observations
per experimental unit are produced, also (generalized) linear mixed models can benefit from 
digital sample size enhancement \cite{kumle_estimating_2021}.

In summary, the adoption of digital tools as per EPPO PP 1/333(1) standard
presents a substantial leap forward in phytosanitary product efficacy and
selectivity trials. By enabling the collection of significantly more data
points, these technologies enhance the robustness of statistical analyses and
support more reliable decision-making in agricultural practices.

\section{Geomatic Technics}
\subsection{Photogrammetry}

Photogrammetry is a geomatic technique that allows the acquisition of spatial
data by processing and analyzing photographic images. It involves capturing a
series of overlapping photos from various perspectives, often using aerial
platforms such as drones or UAVs (Unmanned Aerial Vehicles), and then processing
them using specialized software to generate precise three-dimensional (3D)
models and orthomosaics. In the context of EPPO PP 1/333(1), photogrammetry
offers the ability to obtain high-resolution spatial data, which can greatly
enhance the precision of efficacy and selectivity assessments.

The key benefits of photogrammetry include:

\begin{itemize}
\item \textbf{High Spats correct distortions and
generate georeferenced maps, minimizing errors related to manual measurements.
\item \textbf{Integration with GIS}: Facilitates spatial analysis and
interpretation within Geographic Information Systems.
\end{itemize}

When integrated into efficacy trials, photogrammetry significantly enhances the
ability to detect spatial heterogeneities and assess treatment performance across
large areas. The increased data density and spatial accuracy ultimately lead to
improved statistical power, especially when combined with geostatistical
techniques.

\subsection{Geostatistics}

Geostatistics encompasses a range of statistical methods aimed at analyzing
spatial and spatiotemporal data. In the context of phytosanitary product trials,
geostatistical methods allow for the quantification of spatial variation and the
prediction of treatment effects across plots. The most prominent techniques
include kriging, spatial covariates modeling, and spatial analysis of field trials
with splines.

\textbf{Kriging} is an advanced interpolation method that provides best linear
unbiased predictions (BLUP) of spatially correlated variables. By modeling the
spatial covariance structure through variograms, kriging allows for accurate
prediction of unmeasured locations, reducing spatial noise and enhancing the
interpretation of treatment effects.

\textbf{Spatial Covariates} involve incorporating additional spatial information
(such as soil properties or topography) into statistical models, thereby
improving the accuracy of predictions. Integrating spatial covariates helps to
account for environmental variability and isolate the true treatment effect.

\textbf{Spatial Analysis of Field Trials with Splines} leverages spline functions
to model smooth spatial trends, effectively addressing spatial heterogeneity.
Splines offer a flexible and efficient way to account for gradual changes across
the field, complementing other geostatistical methods.

Combining photogrammetry and geostatistics as outlined in EPPO PP 1/333(1) allows
researchers to gain comprehensive spatial insights, ultimately strengthening the
reliability and validity of efficacy assessments.ial Resolution}: Captures detailed spatial variations
across plots, enabling accurate characterization of treatment effects.
\item \textbf{Time Efficiency}: Rapid data acquisition compared to manual
methods, allowing for frequent monitoring and real-time data analysis.
\item \textbf{Enhanced Accuracy}: Advanced algorithm

\section{Machine Learning}

Machine Learning (ML) is a branch of artificial intelligence (AI) that 
focuses on the development of algorithms and models capable of learning 
from and making predictions or decisions based on data. Unlike traditional 
programming, where explicit instructions dictate the output for given inputs, 
ML models identify patterns and relationships within data to generate 
predictive outcomes. These techniques are particularly valuable when 
dealing with large, complex, or high-dimensional datasets, where manual 
analysis would be impractical or inefficient.

ML has gained substantial importance in various scientific fields, including 
agriculture and plant protection. Within the context of phytosanitary product 
efficacy evaluation, ML offers new opportunities to enhance data processing, 
interpretation, and decision-making by leveraging vast amounts of observational 
data collected during field trials. Integrating ML approaches into the framework 
of PP1/333 can significantly increase the robustness and accuracy of the analysis, 
allowing for more data-driven and automated assessments.

The primary objective of employing ML techniques in phytosanitary product trials 
is to improve accuracy, precision, and reproducibility while reducing manual 
intervention and subjective bias. Modern ML methods can analyze complex interactions 
between variables and predict treatment outcomes under various conditions, thereby 
facilitating more efficient and accurate efficacy assessments.

\subsection{Approaches}

There are several fundamental approaches in machine learning, each suited to different 
types of tasks and data structures:

\begin{itemize}
    \item \textbf{Supervised Learning}: Models are trained on labeled datasets where 
    the input-output relationship is known. Techniques include regression, classification, 
    and ensemble methods such as Random Forests and Gradient Boosting.
    \item \textbf{Unsupervised Learning}: Models identify patterns or groupings within 
    data without labeled responses. Clustering (e.g., K-means, hierarchical clustering) 
    and dimensionality reduction (e.g., PCA, t-SNE) are common techniques.
    \item \textbf{Semi-supervised Learning}: Combines a small amount of labeled data with 
    a large amount of unlabeled data to improve learning accuracy.
    \item \textbf{Reinforcement Learning}: Agents learn by interacting with an environment 
    and receiving feedback in the form of rewards or penalties.
    \item \textbf{Deep Learning}: Utilizes neural networks with multiple layers 
    (deep architectures) to model complex relationships and patterns, particularly 
    in image and signal processing.
\end{itemize}

ML models can also be integrated with statistical techniques, providing hybrid approaches that 
combine inferential statistics with predictive modeling. For example, generalized linear models 
(GLMs) can be enhanced with ML techniques to improve their accuracy and adaptability.

\subsection{Computer Vision}

Computer vision is a subfield of ML that focuses on enabling machines to interpret and analyze 
visual information. In the context of phytosanitary product efficacy evaluation, computer vision 
methods are increasingly used for automated observation and measurement, particularly when 
integrated with digital imaging and photogrammetry.

The use of computer vision within PP1/333 trials significantly enhances data acquisition by 
enabling continuous monitoring and precise measurement of crop conditions. Techniques such as 
image segmentation, object detection, and texture analysis can automatically identify plant 
stress, disease symptoms, and pest damage. Convolutional Neural Networks (CNNs) and related 
architectures, including ResNet and U-Net, have shown high efficacy in analyzing complex 
agricultural images.

Moreover, combining computer vision with geostatistical methods allows for the spatial mapping 
of efficacy across field plots, generating comprehensive visual assessments that support 
statistical evaluations. This integrated approach maximizes the utility of both spatial 
and temporal data, facilitating more robust and accurate assessments while minimizing human 
error and labor requirements.

\chapter{Thesis Aims and Framework: A New Statistical Analysis Workflow}

\chapter{Study Cases}
\section{Continuous Variables}
\subsection{Plant Count}

\section{Ordinal and Nominal Variables}
\subsection{Phytoxicity Score}

\section{Binary Variables}
\subsection{Embedding Spaces for Control Sample Anomaly Detection}

\bibliographystyle{plainnat}
\bibliography{Phd_Thesis_SBumbaca}

\end{document}
