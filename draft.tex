\ref{tab:data_types}.

\begin{table}[ht]
\caption{Types of Data in Phytosanitary Product Evaluation and Appropriate Statistical Approaches}
\label{tab:data_types}
\centering
\begin{tabular}{|p{2.5cm}|p{4cm}|p{5cm}|p{3cm}|}
\hline
\textbf{Data Type} & \textbf{Description} & \textbf{Appropriate Statistical Methods} & \textbf{Examples} \\
\hline
Continuous & Measured on a continuous scale with potentially infinite values & Analysis of variance (ANOVA), linear regression, t-tests, mixed models & Yield measurements, plant height, weight \\
\hline
Discrete (Counts) & Whole number values representing counts & Poisson regression, negative binomial models, log-transformed ANOVA & Insect counts, number of diseased plants \\
\hline
Proportions & Percentages or ratios between 0 and 1 & Beta regression, arcsine-transformed ANOVA, logistic regression & Disease severity percentages, control efficacy \\
\hline
Ordinal & Ranked data with clear ordering & Non-parametric methods (Kruskal-Wallis, Mann-Whitney), ordinal regression & Phytotoxicity scores, development stages \\
\hline
Nominal & Categorical data without natural ordering & Chi-square tests, Fisher's exact test & Color changes, presence of specific symptoms \\
\hline
Binary & Data with only two possible outcomes & Logistic regression, chi-square tests, Fisher's exact test & Presence/absence of disease \\
\hline
\end{tabular}
\end{table}

EPPO guidelines establish a logical sequence for analysis: first determining if 
the trial provides realistic and useful data by checking for sufficient pest infestation 
without excessive variability; then confirming result coherence by verifying that 
reference products perform as expected; and finally, if these conditions are met, 
comparing test products to reference standards and potentially to each other. The 
focus should be on estimating treatment effect magnitudes with appropriate measures 
of statistical uncertainty.